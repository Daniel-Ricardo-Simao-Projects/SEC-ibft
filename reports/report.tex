%%%%%%%%%%%%%%%%%%%%%%%%%%%%%%%%%%%%%%%%%%%%%%%%%%%%%%%%%%%%%%%%%%%%%%%%%%%%%%%%
% Template for USENIX papers.
%
% History:
%
% - TEMPLATE for Usenix papers, specifically to meet requirements of
%   USENIX '05. originally a template for producing IEEE-format
%   articles using LaTeX. written by Matthew Ward, CS Department,
%   Worcester Polytechnic Institute. adapted by David Beazley for his
%   excellent SWIG paper in Proceedings, Tcl 96. turned into a
%   smartass generic template by De Clarke, with thanks to both the
%   above pioneers. Use at your own risk. Complaints to /dev/null.
%   Make it two column with no page numbering, default is 10 point.
%
% - Munged by Fred Douglis <douglis@research.att.com> 10/97 to
%   separate the .sty file from the LaTeX source template, so that
%   people can more easily include the .sty file into an existing
%   document. Also changed to more closely follow the style guidelines
%   as represented by the Word sample file.
%
% - Note that since 2010, USENIX does not require endnotes. If you
%   want foot of page notes, don't include the endnotes package in the
%   usepackage command, below.
% - This version uses the latex2e styles, not the very ancient 2.09
%   stuff.
%
% - Updated July 2018: Text block size changed from 6.5" to 7"
%
% - Updated Dec 2018 for ATC'19:
%
%   * Revised text to pass HotCRP's auto-formatting check, with
%     hotcrp.settings.submission_form.body_font_size=10pt, and
%     hotcrp.settings.submission_form.line_height=12pt
%
%   * Switched from \endnote-s to \footnote-s to match Usenix's policy.
%
%   * \section* => \begin{abstract} ... \end{abstract}
%
%   * Make template self-contained in terms of bibtex entires, to allow
%     this file to be compiled. (And changing refs style to 'plain'.)
%
%   * Make template self-contained in terms of figures, to
%     allow this file to be compiled. 
%
%   * Added packages for hyperref, embedding fonts, and improving
%     appearance.
%   
%   * Removed outdated text.
%
%%%%%%%%%%%%%%%%%%%%%%%%%%%%%%%%%%%%%%%%%%%%%%%%%%%%%%%%%%%%%%%%%%%%%%%%%%%%%%%%

\documentclass[letterpaper,twocolumn,10pt]{article}
\usepackage{usenix2019_v3}

% to be able to draw some self-contained figs
\usepackage{tikz}
\usepackage{amsmath}

% inlined bib file
\usepackage{filecontents}

%-------------------------------------------------------------------------------
\begin{filecontents}{\jobname.bib}
%-------------------------------------------------------------------------------
@Paper{ibft-paper,
  author =       {Moniz H.},
  title =        {The Istanbul BFT Consensus Algorithm},
  year =         2020,
  edition =      {1.00},
  note =         {\url{https://arxiv.org/pdf/2002.03613.pdf}}
}
\end{filecontents}

%-------------------------------------------------------------------------------
\begin{document}
%-------------------------------------------------------------------------------

%don't want date printed
\date{}

% make title bold and 14 pt font (Latex default is non-bold, 16 pt)
\title{\Large \bf HDS Serenity Ledger}

%for single author (just remove % characters)
\author{
{\rm Daniel Pereira}\\
99194
\and
{\rm Ricardo Toscanelli}\\
99315
\and
{\rm Simão Gato}\\
99328
% copy the following lines to add more authors
% \and
% {\rm Name}\\
%Name Institution
} % end author

\maketitle

%-------------------------------------------------------------------------------
\begin{abstract}
%-------------------------------------------------------------------------------
Your abstract text goes here. Just a few facts. Whet our appetites.
Not more than 200 words, if possible, and preferably closer to 150.
ONLY IF WE HAVE SPACE
\end{abstract}


%-------------------------------------------------------------------------------
\section{System Design Overview}
%-------------------------------------------------------------------------------

Our highly dependable system follows a layered architecture designed for scalability and fault tolerance. Let's delve into the key components:

\begin{enumerate}
    \item \textbf{Client Application Interface:} This user-facing layer acts as the entry point for user interactions. It captures user requests and transmits them securely to the Client Service. 

    \item \textbf{Client Service:} This background service acts as a mediator between the Client Application and the server-side logic. It receives requests from the Client Application, performs signature validation (using cryptographic techniques), and broadcasts the message to the relevant server-side service.  

    \item \textbf{SerenityLedgerService:} This core server-side service receives messages broadcasted by the Client Service. It acts as the central coordinator, orchestrating the overall message processing flow. It prepares the data based on the received message and interacts with the Node Service to retrieve the consensus value.

    \item \textbf{Node Service:} This specialized service encapsulates the logic for reaching consensus on a specific value. It interacts with SerenityLedgerService to receive the prepared data and leverages a defined consensus mechanism (Instambul Byzantine Fault Tolerance) to reach an agreement with other nodes. If consensus is achieved, the resulting ledger state is returned to the SerenityLedgerService.

    \item \textbf{Communication Flow:}  The communication primarily follows a client-server model. User interactions initiate at the Client Application, which transmits requests to the Client Service. The Client Service broadcasts the message to the designated SerenityLedgerService on the server side. SerenityLedgerService then interacts with the Node Service to reach consensus on a value. Finally, the agreed-upon ledger state is potentially relayed back to the Client Service for further processing or user notification.
\end{enumerate}

This layered architecture promotes modularity and separation of concerns. Each layer has a well-defined responsibility, improving maintainability and promoting easier integration of future functionalities. The use of a dedicated Node Service for consensus allows for flexibility in exploring different consensus algorithms depending on the specific needs of the system.

%-------------------------------------------------------------------------------
\section{Relevant Implementation Aspects}
%-------------------------------------------------------------------------------

The most relevant implementation aspects; cite something like \cite{ibft-paper}.

%-------------------------------------------------------------------------------
\section{Behavior under attack}
%-------------------------------------------------------------------------------

The experimental evaluation that shows that the system is dependable in the
presence of Byzantine behavior.

%-------------------------------------------------------------------------------
\bibliographystyle{plain}
\bibliography{\jobname}

%%%%%%%%%%%%%%%%%%%%%%%%%%%%%%%%%%%%%%%%%%%%%%%%%%%%%%%%%%%%%%%%%%%%%%%%%%%%%%%%
\end{document}
%%%%%%%%%%%%%%%%%%%%%%%%%%%%%%%%%%%%%%%%%%%%%%%%%%%%%%%%%%%%%%%%%%%%%%%%%%%%%%%%

%%  LocalWords:  endnotes includegraphics fread ptr nobj noindent
%%  LocalWords:  pdflatex acks