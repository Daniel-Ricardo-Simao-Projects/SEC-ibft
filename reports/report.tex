%%%%%%%%%%%%%%%%%%%%%%%%%%%%%%%%%%%%%%%%%%%%%%%%%%%%%%%%%%%%%%%%%%%%%%%%%%%%%%%%
% Template for USENIX papers.
%
% History:
%
% - TEMPLATE for Usenix papers, specifically to meet requirements of
%   USENIX '05. originally a template for producing IEEE-format
%   articles using LaTeX. written by Matthew Ward, CS Department,
%   Worcester Polytechnic Institute. adapted by David Beazley for his
%   excellent SWIG paper in Proceedings, Tcl 96. turned into a
%   smartass generic template by De Clarke, with thanks to both the
%   above pioneers. Use at your own risk. Complaints to /dev/null.
%   Make it two column with no page numbering, default is 10 point.
%
% - Munged by Fred Douglis <douglis@research.att.com> 10/97 to
%   separate the .sty file from the LaTeX source template, so that
%   people can more easily include the .sty file into an existing
%   document. Also changed to more closely follow the style guidelines
%   as represented by the Word sample file.
%
% - Note that since 2010, USENIX does not require endnotes. If you
%   want foot of page notes, don't include the endnotes package in the
%   usepackage command, below.
% - This version uses the latex2e styles, not the very ancient 2.09
%   stuff.
%
% - Updated July 2018: Text block size changed from 6.5" to 7"
%
% - Updated Dec 2018 for ATC'19:
%
%   * Revised text to pass HotCRP's auto-formatting check, with
%     hotcrp.settings.submission_form.body_font_size=10pt, and
%     hotcrp.settings.submission_form.line_height=12pt
%
%   * Switched from \endnote-s to \footnote-s to match Usenix's policy.
%
%   * \section* => \begin{abstract} ... \end{abstract}
%
%   * Make template self-contained in terms of bibtex entires, to allow
%     this file to be compiled. (And changing refs style to 'plain'.)
%
%   * Make template self-contained in terms of figures, to
%     allow this file to be compiled. 
%
%   * Added packages for hyperref, embedding fonts, and improving
%     appearance.
%   
%   * Removed outdated text.
%
%%%%%%%%%%%%%%%%%%%%%%%%%%%%%%%%%%%%%%%%%%%%%%%%%%%%%%%%%%%%%%%%%%%%%%%%%%%%%%%%

\documentclass[letterpaper,twocolumn,10pt]{article}
\usepackage{usenix2019_v3}

% to be able to draw some self-contained figs
\usepackage{tikz}
\usepackage{amsmath}

% inlined bib file
\usepackage{filecontents}

%-------------------------------------------------------------------------------
\begin{filecontents}{\jobname.bib}
%-------------------------------------------------------------------------------
@Paper{ibft-paper,
  author =       {Moniz H.},
  title =        {The Istanbul BFT Consensus Algorithm},
  year =         2020,
  edition =      {1.00},
  note =         {\url{https://arxiv.org/pdf/2002.03613.pdf}}
}
\end{filecontents}

%-------------------------------------------------------------------------------
\begin{document}
%-------------------------------------------------------------------------------

%don't want date printed
\date{}

% make title bold and 14 pt font (Latex default is non-bold, 16 pt)
\title{\Large \bf HDS Serenity Ledger}

%for single author (just remove % characters)
\author{
{\rm Daniel Pereira}\\
99194
\and
{\rm Ricardo Toscanelli}\\
99315
\and
{\rm Simão Gato}\\
99328
% copy the following lines to add more authors
% \and
% {\rm Name}\\
%Name Institution
} % end author

\maketitle

%-------------------------------------------------------------------------------
\begin{abstract}
%-------------------------------------------------------------------------------
DO ABSTRACT ONLY IF WE HAVE SPACE FOR IT
\end{abstract}


%-------------------------------------------------------------------------------
\section{System Design Overview}
%-------------------------------------------------------------------------------

Our system follows a layered architecture designed for scalability and fault tolerance. The key components are:

\begin{enumerate}
    \item \textbf{Client Application Interface:} This user-facing layer acts as the entry point for user interactions. It captures user requests and transmits them securely to the Client Service. 

    \item \textbf{Client Service:} This background service acts as a mediator between the Client Application and the server-side logic. It receives requests from the Client Application, performs signature validation (using cryptographic techniques), and broadcasts the message to the relevant server-side service.  

    \item \textbf{SerenityLedgerService:} This core server-side service receives messages broadcasted by the Client Service. It acts as the central coordinator, orchestrating the overall message processing flow. It prepares the data based on the received message and interacts with the Node Service to retrieve the consensus value.

    \item \textbf{Node Service:} This specialized service encapsulates the logic for reaching consensus on a specific value. It interacts with SerenityLedgerService to receive the prepared data and leverages a defined consensus mechanism (Instambul Byzantine Fault Tolerance) to reach an agreement with other nodes. If consensus is achieved, the resulting ledger state is returned to the SerenityLedgerService.

    \item \textbf{Communication Flow:}  The communication primarily follows a client-server model. User interactions initiate at the Client Application, which transmits requests to the Client Service. The Client Service broadcasts the message to the designated SerenityLedgerService on the server side. SerenityLedgerService then interacts with the Node Service to reach consensus on a value. Finally, the agreed-upon ledger state is potentially relayed back to the Client Service for further processing or user notification.
\end{enumerate}

This layered architecture promotes modularity and separation of concerns. Each layer has a well-defined responsibility, improving maintainability and promoting easier integration of future functionalities. The use of a dedicated Node Service for consensus allows for flexibility in exploring different consensus algorithms depending on the specific needs of the system.

%-------------------------------------------------------------------------------
\section{Relevant Implementation Aspects}
%-------------------------------------------------------------------------------

This section highlights some key implementation aspects of our system:

\subsection{IBFT}
We leverage the Istanbul BFT (IBFT) protocol for consensus, accordingly to the official IBFT paper \cite{ibft-paper}, ensuring fault tolerance even in the presence of Byzantine failures (nodes exhibiting arbitrary behavior). The implementation has been thoroughly reviewed and corrected since the initial stages of the project.

\subsection{Triggered Consensus}
The IBFT protocol execution is optimized to initiate only when a specific number (X) of requests are queued. This reduces unnecessary consensus rounds and improves system efficiency. During the commit phase upon reaching a "DECIDE" state, the participating nodes process the requests accumulated in the queu, by removing them from the queue and executing them.

\subsection{Parallel Consensus}
Our system enables concurrent execution of multiple IBFT consensus instances. However, to maintain data consistency, ongoing instance Y waits for the completion of instance X (where X precedes Y) if transactions within Y depend on the outcome of transactions in X. This parallel execution can introduce potential complexities if such dependencies exist between concurrently processed requests. 

\subsection{Ledger Design}
The system maintains a ledger consisting of a blockchain structure where each block stores a predefined number (X) of transactions. Additionally, the ledger holds the current state of all accounts within the system.

\subsection{Client-Side Verification}
When a client queries their account balance, they must receive confirmation from 2f+1 nodes to ensure the retrieved value's accuracy. This requirement arises due to the presence of account states within the ledger. While using a system like UTXO could potentially offer advantages, time constraints during development prevented its implementation.

\subsection{Meeting Application Requirements}

Our system design effectively meets the following application requirements:

\begin{itemize}
\item \textbf{Non-Negative Account Balances:} The system enforces non-negative account balances by rejecting any transaction attempting to spend more money than the account's current balance. Clients attempting such transfers will receive an error message.

\item \textbf{Account State Protection:} Unauthorized users cannot modify account states. Each transaction requires a digital signature, ensuring only the legitimate owner of an account can initiate transfers. This digital signature acts as an authorization mechanism, preventing unauthorized access and modifications.

\item \textbf{Non-Repudiation of Operations:} We achieve non-repudiation through a combination of digital signatures and nonces. Each transaction is accompanied by a digital signature that binds the operation to the account owner. Additionally, we use a nonce, which is an incrementing number for each request sent by a client. The server verifies that the received nonce is greater than any previously received nonce for the same client. This prevents replay attacks and ensures that the client cannot repudiate a previously authorized transaction.
\end{itemize}

These features, combined with the Byzantine Fault Tolerance provided by the IBFT protocol, guarantee the integrity and security of the system's state, upholding the required application-level properties.

\subsection{Transaction Fees and Incentive Design}

This subsection addresses the introduction of transaction fees within our system, a common feature in modern blockchains. Transaction fees serve a dual purpose:

\begin{itemize}
    \item \textbf{Discourage Spam:} A small fee per transaction discourages malicious actors from flooding the network with irrelevant transactions, thereby protecting system resources and maintaining network efficiency. This concept is similar to "gas" fees used in blockchains like Ethereum.
    \item \textbf{Incentivize Block Production:} Transaction fees provide an economic incentive for nodes to participate in the consensus process (e.g., block production in Proof-of-Work systems or leader election in IBFT). These fees reward nodes for the computational power and resources they contribute to secure the network.
\end{itemize}

\subsubsection*{Fee Structure Considerations} While the specific fee structure chosen for a system can be flexible, we decided to use a fee percentage of 5\% because this is a small environment compared to the real blockchain systems. A common approach is to set the transaction fee as a percentage of the transferred value. A more balanced approach could be a fixed fee (e.g., 1 unit of the coin) for simple transactions and a variable fee based on transaction size or computational complexity for more resource-intensive operations. This aligns with the fee structure used in blockchains like Solana, where fees depend on the computational resources required for transaction processing.

\subsubsection*{Fee Distribution} The collected transaction fees can be distributed in various ways. One option is to allocate them entirely to the leader node responsible for processing the block. Alternatively, a portion could be distributed among participating nodes in the consensus process, further incentivizing their involvement. In our system implementation, no node receives the fee, so the cash just "disappears". Would need to be corrected in the future.

\subsubsection*{Justification and Trade-offs} 
The chosen fee structure needs to balance several factors:

\begin{itemize}
    \item Discouraging spam requires a sufficient fee to make it economically unviable for malicious actors.
    \item Keeping fees too low might lead to network congestion.
    \item Setting fees too high could hinder user adoption and limit system usability.
\end{itemize}

%-------------------------------------------------------------------------------
\section{Behavior under attack}
%-------------------------------------------------------------------------------

This section evaluates our system's dependability under Byzantine faults, where nodes (clients or servers) exhibit arbitrary and potentially malicious behavior.

\subsection{Experimental Setup}

We leveraged the configuration JSON file to introduce Byzantine behavior. Both clients and servers had a dedicated field within the JSON configuration specifying their Byzantine behavior mode.

\subsection{Client-side Byzantine Faults} We explored various client-side Byzantine behaviors, including:
    \begin{itemize}
        \item Arbitrary spending: Clients attempted to spend more money than their account balance.
        \item Double spending: Clients tried to spend the same funds twice.
        \item Self-payment: Clients attempted to transfer funds from one account to itself disguised as another account.
    \end{itemize}

For all these scenarios, our system successfully prevented the malicious client from manipulating the ledger, ensuring overall system correctness. Most client-side Byzantine fault tests were conducted with two clients, as we deemed this sufficient to cover various attack vectors.

\subsection{Server-side Byzantine Faults} We investigated a broader range of Byzantine behaviors for servers/nodes. Most tests involved a system with four nodes, allowing for a maximum of one faulty node. The explored server-side Byzantine behaviors included:
    \begin{itemize}
        \item Forged leadership: A node falsely claimed to be the leader, attempting to disrupt the consensus process.
        \item Value alteration: Nodes tampered with the consensus value within messages.
        \item Message spoofing: Nodes impersonated other nodes to trick the system into accepting fabricated messages and achieve a quorum.
        \item Incorrect value transmission: Nodes deliberately sent wrong values to users.
    \end{itemize}

Our system successfully reached consensus in all server-side Byzantine fault tests. However, a significant performance difference was observed between scenarios requiring "Round Changes" and those that did not. Round Changes, a mechanism to maintain system liveness under faults, substantially increased the time required to reach consensus. While this ensures system progress even in challenging situations, it comes at the cost of reduced efficiency.

\subsection{Discussion}

The experimental evaluation demonstrates the effectiveness of our system in handling various Byzantine faults from both clients and servers. The system successfully prevented malicious actors from manipulating the ledger and disrupting overall system operation. The observed performance impact of Round Changes highlights a trade-off between liveness and efficiency, which can be further optimized in future iterations.

%-------------------------------------------------------------------------------
\bibliographystyle{plain}
\bibliography{\jobname}

%%%%%%%%%%%%%%%%%%%%%%%%%%%%%%%%%%%%%%%%%%%%%%%%%%%%%%%%%%%%%%%%%%%%%%%%%%%%%%%%
\end{document}
%%%%%%%%%%%%%%%%%%%%%%%%%%%%%%%%%%%%%%%%%%%%%%%%%%%%%%%%%%%%%%%%%%%%%%%%%%%%%%%%

%%  LocalWords:  endnotes includegraphics fread ptr nobj noindent
%%  LocalWords:  pdflatex acks